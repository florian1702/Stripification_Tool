\chapter{Weitere Anpassungen}

In diesem Kapitel werden die Änderungen an weiteren Code-Stellen erläutert.
\\
\begin{itemize}
    \item 
    In der \lstinline{main.cpp} wurden Anpassungen vorgenommen, um die
    Callback-Funktionen für das Fenster-Management zu aktualisieren und die
    Interaktionen zwischen ImGui und der Hauptfensterinstanz zu ermöglichen. Die
    Funktionen \lstinline{reshape}, \lstinline{keyboard}, \lstinline{mouse} und
    \lstinline{move} wurden so modifiziert, dass sie sowohl die
    ImGui-spezifischen Anforderungen erfüllen als auch die entsprechenden
    Methoden des ViewerWindow-Objekts aufrufen. 
    \\
    Zusätzlich wurde ImGui in der
    \lstinline{main()}-Funktion initialisiert, indem die ImGui-Version
    überprüft, ein ImGui-Kontext erstellt, das Dark-Theme angewendet und die
    ImGui-Backends für GLUT und OpenGL3 konfiguriert wurden.
    \\
    \item 
    In der \lstinline{CGlutWindow}-Klasse wurden spezifische Anpassungen für die Integration
    von ImGui vorgenommen. Beim Destruktor der Klasse wurde der Code hinzugefügt, um ImGui-Ressourcen zu
    bereinigen.
    \\
    Zusätzlich wurde in der \lstinline{renderFrame()}-Methode der Code ergänzt, um einen
    neuen ImGui-Frame zu beginnen und zu rendern. 
    \\
    \item 
    Alle Funktionalitäten zum Einlesen und Verarbeiten von Normalen und
    Texturkoordinaten wurden entfernt, da sie nicht mehr benötigt werden.
    \\
    \item 
    Entsprechende Header-Dateien wurden angepasst, um die neuen Funktionalitäten zu integrieren.
\end{itemize}