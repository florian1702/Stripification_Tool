\chapter{Einleitung}

Das Ziel dieses Projekts ist es, nach dem Einlesen einer OBJ-Datei, die
enthaltenen Dreiecke in Strips zusammenzufassen und diese effizient zu zeichnen.
Die Technik der Stripification spielt dabei eine zentrale Rolle und bringt
signifikante Vorteile mit sich.
\\
\\
Stripification bezeichnet den Prozess, bei dem Dreiecke eines 3D-Modells in
sogenannte Triangle Strips umgewandelt werden. Ein Triangle Strip ist eine
Sequenz von Vertices, bei der jedes aufeinanderfolgende Dreieck durch Hinzufügen
eines weiteren Vertex definiert wird. Diese Methode reduziert den Speicherbedarf
für die Vertices auf der GPU, da weniger redundante Daten gespeichert werden
müssen.Beispielsweise wird aus den einzelnen Dreiecken mit Vertices (1, 2, 3),
(3, 2, 4) und (3, 4, 5) nach der Stripification der Strip (1, 2, 3, 4, 5). Dies
bedeutet, dass die Vertices 3 und 4 nicht mehrfach gespeichert werden müssen,
was die Effizienz der Speicher- und Renderprozesse erhöht. 
\\
\\
Das Projekt baut auf dem bestehenden Projekt „ObjToVertexBuffer“ auf, das in
einer Übungsstunde bearbeitet wurde. Dieses Ausgangsprojekt stellt bereits einen
OBJ-Datei-Parser zum Einlesen von 3D-Modellen sowie einen fertigen Canvas mit
den passenden Datenstrukturen bereit, was eine solide Basis für die
Weiterentwicklung darstellt.
\\
\\
Zur Erreichung der Projektziele wurden folgende Teilaufgaben definiert und
umgesetzt:
\begin{itemize}
    \item Implementierung des Triangels-to-Strip Konverters
    \item Implementierung des Strip-Zeichners
    \item Allgemeine Anpassung des vorhandenen Codes
    \item Implementierung der Steuerung und einer GUI
\end{itemize}
\hfil \break
Die von mir ergänzten Codeabschnitte sind zur besseren Nachverfolgbarkeit mit
Kommentaren markiert:

\begin{lstlisting}
    // <Florian>
    ...
    // </Florian>
\end{lstlisting}