\chapter{Implementierung des Strip-Zeichners}

Der Code, der für das Zeichnen von Strips zuständig ist, befindet sich in der
Methode \lstinline{void Mesh3D::draw()} in der Datei \lstinline{Mesh3d_render.cpp}. Diese Methode
verarbeitet und zeichnet die Strips der 3D-Mesh-Daten, die in der Mesh3D-Klasse
gespeichert sind. Die folgende detaillierte Beschreibung erklärt die
Funktionsweise dieser Methode.
\\
\\
Zu Beginn der Methode wird der Zufallszahlengenerator der Standardbibliothek
initialisiert. Dazu wird der Seed des Generators auf 0 gesetzt, um
sicherzustellen, dass die Farben bei jedem Aufruf der Methode \lstinline{draw()} konsistent
und wiederholbar sind. Dies bedeutet, dass trotz der zufälligen Natur der
Farbwerte die gleiche Farbe für denselben Strip bei jedem Methodenaufruf
verwendet wird. 
\\
\\
Es wird eine Zählvariable \lstinline{int counter} deklariert und auf 0 initialisiert. Diese
Variable dient dazu, die Anzahl der gezeichneten Strips während der
Schleifeniterationen zu zählen. Diese Zählung wird verwendet, um die maximale
Anzahl der zu zeichnenden Strips zu begrenzen, die durch die Konstante
\lstinline{strip_amount_limit} definiert ist.
\\
\\
Die Methode beginnt dann mit einer äußeren Schleife, die über alle Strip-Daten
iteriert. Der Schleifenmechanismus bricht ab, wenn die Anzahl der gezeichneten
Strips die festgelegte Grenze (\lstinline{strip_amount_limit}) erreicht. Dies stellt sicher,
dass nicht mehr Strips gezeichnet werden als zulässig.
\\
\\
Innerhalb der Schleife wird eine zufällige Farbe für den aktuellen Strip
generiert. Dies erfolgt durch Erzeugung von drei float-Werten \lstinline{r}, \lstinline{g} und \lstinline{b}, die
die RGB-Farben repräsentieren. Diese Werte werden durch die Division der vom
Zufallszahlengenerator zurückgegebenen Werte durch \lstinline{RAND_MAX} berechnet. Dies
sorgt dafür, dass die Werte im Bereich [0, 1] liegen, was der Farbdarstellung in
OpenGL entspricht. Die Funktion \lstinline{glColor3f(r, g, b)} wird verwendet, um diese
Farbe für den aktuellen Strip festzulegen.
\\
\\
Anschließend wird der Strip gezeichnet. Dies beginnt mit dem Aufruf von \break
\lstinline{glBegin(GL_TRIANGLE_STRIP)}, welcher OpenGL mitteilt, dass ein neuer Strip
gezeichnet wird. Innerhalb einer inneren Schleife werden die Indizes des
aktuellen Strips durchlaufen. Für jeden Index wird der entsprechende Vertex aus
dem \lstinline{vertices}-Array geladen und dessen Koordinaten an die GPU übergeben, indem
\lstinline{glVertex3f(float x, float y, float z)} aufgerufen wird.
\\
\\
Sobald alle Vertices für den aktuellen Strip verarbeitet wurden, wird die innere
Schleife beendet, und mit \lstinline{glEnd()} wird der Strip abgeschlossen und gezeichnet.
Die Zählvariable \lstinline{counter} wird inkrementiert, und die äußere Schleife fährt mit
dem nächsten Strip fort, bis entweder alle Strips gezeichnet wurden oder die
Grenze \lstinline{strip_amount_limit} erreicht ist.
\\
\\
Zusammengefasst, bereitet die Methode \lstinline{draw()} die Farbwerte vor, verwendet
OpenGL-Funktionen zum Zeichnen der Strips und sorgt durch die Zählvariable
dafür, dass nur eine begrenzte Anzahl von Strips gezeichnet wird. Die Verwendung
eines festen Seeds für den Zufallszahlengenerator sorgt dafür, dass die Farben
konsistent bleiben, während der Code in der Schleife die Strips effizient an die
GPU sendet und zeichnet.