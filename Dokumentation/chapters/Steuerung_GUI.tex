\chapter{Implementierung der Steuerung und der GUI}

Die Methode \lstinline{renderGui()} und die Methode \lstinline{keyEvent(unsigned char key, int x, int y)} 
sind zentrale Bestandteile der Benutzeroberfläche und Steuerung für das
\lstinline{ViewerWindow}-Objekt. Im Folgenden wird detailliert beschrieben, wie diese
Methoden implementiert sind und welche Aufgaben sie erfüllen.
\\
\\
Die Methode \lstinline{renderGui()} ist verantwortlich für die Darstellung und
Aktualisierung der grafischen Benutzeroberfläche unter Verwendung von
ImGui, einer weit verbreiteten und leistungsfähigen Bibliothek für sofortige
GUI-Entwicklung (siehe \href{https://github.com/ocornut/imgui}{ImGui}). Diese Methode verwaltet die Benutzeroberfläche,
die dem Nutzer ermöglicht, verschiedene Interaktionen und Einstellungen
vorzunehmen.
\\
\begin{enumerate}
    \item \textbf{Ladefenster:}
    \\
    Wenn die Variable \lstinline{isLoading} auf true gesetzt ist, zeigt die Methode
    ein Ladefenster an, das den Benutzer darüber informiert, dass eine Aktion (wie
    das Laden eines neuen Meshes) im Gange ist. Dies geschieht durch den Aufruf von
    \lstinline{ImGui::Begin()} mit entsprechenden Flags, um das Fenster ohne Titel, Größe und
    Bewegung anzuzeigen. \lstinline{ImGui::Text()} wird verwendet, um den Ladehinweis
    darzustellen.
    \\
    \item \textbf{Haupt-Benutzeroberfläche:} 
    \\ 
    Sobald \lstinline{isLoading} auf false
    gesetzt ist, zeigt die Methode die Haupt-GUI an, die mehrere Abschnitte
    enthält: 
    \\
    \begin{itemize}
        \item 
        Unter dem Abschnitt 'Control' wird dem Benutzer erklärt, welche
        Tastenkombinationen zur Steuerung der Anzahl der anzuzeigenden Strips
        verwendet werden können. Die Beschreibungen dieser Steuerungen werden
        durch \lstinline{ImGui::Text()} bereitgestellt.
        \\
        \item 
        Im Abschnitt 'Mesh' kann der Benutzer zwischen verschiedenen
        Mesh-Dateien auswählen. Hierzu wird \lstinline{ImGui::BeginCombo()}
        verwendet, um ein Dropdown-Menü zu erstellen. Die Auswahl wird durch
        \lstinline{ImGui::Selectable()} ermöglicht, wobei die Variable
        \lstinline{selectedIndex} die Auswahl speichert. \\
        \item 
        Im Abschnitt 'Info' werden verschiedene Informationen über das aktuelle
        Mesh angezeigt. Dazu gehören die Anzahl der einzigartigen Vertices, der
        Dreiecksflächen und der Strips sowie die Anzahl und Einsparungen von
        Vertices, die an die GPU gesendet wurden. Berechnungen zur Anzahl der
        gesparten Vertices werden durchgeführt, und die Ergebnisse werden
        mittels \lstinline{ImGui::Text()} angezeigt. 
        \\
        \item 
        Ein Button mit der Beschriftung 'Process' ermöglicht das
        Laden eines neuen Meshes. Beim Klicken auf diesen Button wird der Status
        \lstinline{isLoading} auf true gesetzt und ein neuer Thread gestartet, der das Mesh
        aus einer .obj-Datei lädt. Das alte Mesh wird durch das neue ersetzt,
        und \lstinline{isLoading} wird auf false zurückgesetzt, sobald der Ladevorgang
        abgeschlossen ist.
    \end{itemize} 
    \hfil \break
    \item \textbf{Rendering und Abschluss:} 
    \\ 
    Nach der Erstellung und Aktualisierung der GUI wird \lstinline{ImGui::Render()}
    aufgerufen, um die GUI zu rendern, und \lstinline{ImGui_ImplOpenGL3_RenderDrawData()}
    wird verwendet, um die gerenderten GUI-Daten mit OpenGL anzuzeigen.
\end{enumerate}
\hfil \break
Die Methode \lstinline{keyEvent(unsigned char key, int x, int y)} wurde angepasst, um
Tastatureingaben zu verarbeiten und entsprechende Änderungen in der Anzeige der
Strips vorzunehmen. Der Parameter \lstinline{key} gibt die gedrückte Taste an, und basierend
auf diesem Wert wird eine der folgenden Aktionen ausgeführt:
\\
\begin{itemize}
    \item Wenn die Taste '1' gedrückt wird, wird die Anzahl der anzuzeigenden Strips
    um eins reduziert, vorausgesetzt, die Anzahl ist nicht bereits auf 0 reduziert.
    \\
    \item Bei Drücken der Taste '2' wird die Anzahl der anzuzeigenden Strips um eins
    erhöht, sofern sie nicht bereits die maximale Anzahl der verfügbaren Strips
    erreicht hat.
    \\
    \item Wenn die Taste '3' gedrückt wird, wird entweder die Anzahl
    der anzuzeigenden Strips auf 0 gesetzt (was alle Strips ausblendet) oder auf die
    Gesamtanzahl der verfügbaren Strips, um alle anzuzeigen.
\end{itemize}
